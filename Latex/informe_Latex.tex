\documentclass[]{article}
\usepackage[top=3cm, bottom=3cm, outer=3cm, inner=3cm]{geometry}
\usepackage{graphicx} % Required for inserting images
\usepackage{float}
\usepackage{array}
\usepackage{xcolor}
\usepackage{listings}
\usepackage{color, colortbl}
\usepackage{url}

\definecolor{fondo}{rgb}{0.8, 0, 0}


\title{prueba}
\author{JEAN CARLO CHARA CONDORI}
\date{June 2023}
%\pagestyle{myheadings}

\newcommand{\universidad}{Universidad Nacional de San Agustín de Arequipa}
\newcommand{\facultad}{Facultad de Ingeniería de Producción y Servicios}
\newcommand{\departamento}{Departamento Académico de Ingeniería de Sistemas e Informática}
\newcommand{\escuela}{Escuela Profesional de Ingeniería de Sistemas}
\newcommand{\curso}{Programación Web 2}
\newcommand{\estudiante}{Chara Condori Jean Carlo}

% para cabeceras
\usepackage{fancyhdr}
\pagestyle{fancy}

\fancyhf{}
\setlength{\headheight}{30pt}
\renewcommand{\headrulewidth}{1pt}
\renewcommand{\footrulewidth}{1pt}
\fancyhead[L]{\raisebox{-0.2\height}{\includegraphics[width=3cm]{../img/logo_episunsa.png}}}
\fancyhead[C]{\fontsize{7}{7}\selectfont	\universidad \\ \facultad \\ \departamento \\ \escuela \\ \textbf{\curso}}
\fancyhead[R]{\raisebox{-0.2\height}{\includegraphics[width=1.2cm]{../img/logo_abet}}}
\fancyfoot[L]{\estudiante}
\fancyfoot[C]{\curso}
\fancyfoot[R]{Página \thepage}

\newcolumntype{x}[1]{>{\centering\arraybackslash\hspace{0pt}}p{#1}}
% color para tablas


\begin{document}

    \begin{center}	
		\fontsize{17}{17} \textbf{ Informe de Laboratorio 04}
	\end{center}
    \centerline{\textbf{\Large Tema: Python}}

	\begin{table}[H]
		\begin{tabular}{|x{4.7cm}|x{4.8cm}|x{4.8cm}|}
			\hline 
			\rowcolor{fondo}
			\color{white}\textbf{Estudiante} & \color{white}\textbf{Escuela}  & \color{white}\textbf{Asignatura}   \\
			\hline 
			\estudiante & \escuela &  \curso  \\
			\hline 
		\end{tabular}
	\end{table}
    \begin{table}[H]
		\begin{tabular}{|x{4.7cm}|x{4.8cm}|x{4.8cm}|}
			\hline 
			\rowcolor{fondo}
			\color{white}\textbf{Laboratorio} & \color{white}\textbf{Tema}  & \color{white}\textbf{Duracion}   \\
			\hline 
			04 & Python & 04 horas  \\
			\hline 
		\end{tabular}
	\end{table}
    \begin{table}[H]
		\begin{tabular}{|x{4.7cm}|x{4.8cm}|x{4.8cm}|}
			\hline 
			\rowcolor{fondo}
			\color{white}\textbf{Semestre académico} & \color{white}\textbf{Fecha de inicio}  & \color{white}\textbf{Fecha de entrega}   \\
			\hline 
			2023-A & 29 Mayo 2023 & 09 Junio 2023  \\
			\hline 
		\end{tabular}
	\end{table}
 
    \section{Tarea}
	\begin{itemize}		
		\item Para resolver los siguientes ejercicios sólo está permitido usar ciclos, condicionales, definición de listas por
        comprensión, sublistas, map, join, (+), lambda, zip, append,pop, range.
		\item Implemente los metodos de la clase Picture.
        Se recomienda que implemente la clase picture por etapas, probando realizar los dibujos que se muestran en
        las siguientes preguntas.
		\item Usando únicamente los métodos de los objetos de la clase Picture dibuje las siguientes figuras (invoque a draw):
	\end{itemize}
 
    \section{Resolucion}
    	\begin{itemize}		
    		\item 
    		\item 
    		\item 
    	\end{itemize}
    \section{Solucion del Cuestionario}
    	\begin{itemize}		
    		\item 
    		\item 
    		\item 
    	\end{itemize}
    \section{Conclusiones}
    	\begin{itemize}		
    		\item 
    		\item 
    		\item 
    	\end{itemize}
    \section{Retroalimentacion general}
    	\begin{itemize}		
    		\item 
    		\item 
    		\item 
    	\end{itemize}
    \section{Referencias y bibliografias}
    	\begin{itemize}		
    		\item \url{https://docs.python.org/3/library/array.html?highlight=pop#array.array.pop}
    		\item \url{https://docs.python.org/3/tutorial/introduction.html#lists}
    		\item 
            \url{https://docs.python.org/3/glossary.html#term-function}
            \item \url{https://docs.python.org/es/3/tutorial/classes.html}
    	\end{itemize}
     
\end{document}
